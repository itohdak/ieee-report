\section{Maintaining Learned Motions}
In addition to the learning procedure, we considerd a more high level concept; how the robot can store the learned motions. Thinking about the household environment, it is important to enable the human, anytime and anywhere, to require the robot what he/she wants the robot to do. One of the easiest way is by aural instructions. To make it possible, the robot should know the relation between the motions and the natural language words, the names, of them. Although it is difficult for robots to recognize them with no hint, we know them and can teach the robots about them. %% Naming the motions is helpful not only when the human make the robot practice a certain motion, but also when we want to teach the robot when to practice a certain motion associating with its motivating conditions. We integrated the aural communication into the whole learning process.

\subsection{Labeling Motions with Motion Names}
First, the human tell the robot what kind of motion to teach. For example, the human can say ``I will teach you how to push the button.'' The robot can now know that the motion is called ``push the button,'' and label the taught motion with that name. The pair of the motion trajectory data and the motion name are stored in the database. Then the human can ask the robot to do ``push the button'' motion by its name; ``Please push the button.'' The robot can search the database for ``push the button'' motion, and if found, the robot can reproduce the motion according to the learned trajectory data.\par
To extract the name of the motion from what the human say, we utilized Dialogflow, the natural language processing engine. Using the aural commands make the human hand-free and it is the easiest way even for the users who don't know much about the robot.

\subsection{Associating Motions with Motivating Conditions}
Labelling the motions with those indicating names are not only useful when the human orders the robot what he/she wants the robot to do, but also make it easier to associate the motions with those motivating conditions. Some motions should be done whenever the robot experiences some conditions even if there is no human around the robot. These conditions may be visual conditions, aural conditions or someother sensor conditions. To stop the fire when the kettle whistling or to open the door when the door is knocked are some examples of aural conditions.
