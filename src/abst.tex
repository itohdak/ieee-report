This is the document for RINKO in Creative Imformatics Department, The University of Tokyo. My bachelor thesis and my recent works are introduced in this paper. I have been working on cooperative works between a human and a humanoid robot in a household environment. One of the difficulties in such tasks is that the flexibility of executing tasks and the autonomousity of the robot actions have to be ensured at the same time. We tackled this difficulty by decomposing cooperative works into two layers: local action generation and global transitions between those actions. In the former, the robot acquires its motions based on human imitation. In the latter, the robot switches its motion according to the human instructions. The effectiveness of the proposed system is shown in the conducted experiments.

%% Cooperative works between a human and a humanoid robot are important considering household chores. We have constructed a system to realize various cooperative tasks by a human and a humanoid robot. The cooperative works we worked on are those which have to be done by two workers because of the size or the weight of the objects that are too big or heavy to be manipulated by one person (e.g. folding a large tablecloth or carrying a large board). In this paper, we propose a interactive human-robot cooperative task executing system. In the system, cooperative works are decomposed into two layers; local action generation and global transitions between those actions. In the former, the robot acquires its motions based on human imitation and feedback modification applying force and voice. In the latter, the robot switches its motion according to the human instructions. Superiority of this system can be described with the ability in general use due to the sustainability and the flexibility. The effectiveness of the proposed collaborative system is shown in the conducted experiments, executing cooperation tasks that don't have fixed orders between each contained process.

%% In our proposed system, the robot decides its motions by observing human motions and imitating them. By applying force sensor feedback, these motions are modified if there are some constraints on the target objects. In addition to the robot motion generation, we introduced an aural human interface so that a human can order the robot when to show those motions. Eventually, we build the collaborative system that can work continuously by combining visual, haptic and aural information. The conducted experiments show the effectiveness of the proposed collaborative system with executing cooperation tasks that don't have fixed orders between each contained process.

%% This electronic document is a latex template. The various components of your paper [title, text, heads, etc.] are already defined on the style sheet, as illustrated by the portions given in this document.
