\section{Conclusions}
In this study, we worked on the human-humanoid collaborative works manipulating large objects that will be hard to be executed by one. To realize various tasks in a series of household chores, the robot has to acquire flexible motions that are not programmed for a certain task. We took an approach to meet that expectation by taking human motions into account for the robot motions. Moreover, modifying those generated motions with the use of force information and aural communication, both rigid and flexible objects can be manipulated in cooperation with a human. Eventually, enabling those motions be activated selectively and continuously based on the multimodal human instructions, tasks can be finely executed in which the order between the included processes is random.
Our contributions are mainly based on the decomposition of the cooperative works into the local action generation and the global transition framework. The sequential task experiment in this research confirms that this system can perfectly work in human-robot collaborative works in the household environment where the human has the initiative and wants the robot to be an obedient assistant.
