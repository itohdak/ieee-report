\documentclass[letterpaper, 10 pt, conference]{ieeeconf}  % Comment this line out if you need a4paper
%\documentclass[a4paper, 10pt, conference]{ieeeconf}      % Use this line for a4 paper

\IEEEoverridecommandlockouts                              % This command is only needed if 
                                                          % you want to use the \thanks command

\overrideIEEEmargins                                      % Needed to meet printer requirements.

% See the \addtolength command later in the file to balance the column lengths
% on the last page of the document

\usepackage{bm}
\usepackage{url}
\usepackage{siunitx}
\usepackage[dvipdfmx]{graphicx}
\graphicspath{{figs/}}
\usepackage{amsmath}
\usepackage{amssymb}
\usepackage{algorithm}
\usepackage{algpseudocode}
\usepackage{times}
\usepackage[subrefformat=parens,font=footnotesize]{subcaption}
\usepackage{wrapfig}
\captionsetup[figure]{font=footnotesize}
%
\newcommand{\figref}[1]{Fig.\ref{figure:#1}}
\newcommand{\tabref}[1]{Table \ref{table:#1}}
\newcommand{\secref}[1]{Sec.\ref{sec:#1}}
\newcommand{\subsecref}[1]{Sec.\ref{subsec:#1}}
\newcommand{\equref}[1]{Eq.\ref{equation:#1}}
\newcommand{\argmax}{\operatornamewithlimits{argmax}}
\newcommand{\argmin}{\operatornamewithlimits{argmin}}

\title{\LARGE \bf
About My Bachelor Thesis:\\ %% and My Recent Works
Human-Robot Cooperative Task Execution
%% Preparation of Papers for IEEE Sponsored Conferences \& Symposia*
}


\author{Hideaki Ito\\
  The University of Tokyo, JSK Lab
%% \thanks{*This work was not supported by any organization}% <-this % stops a space
%% \thanks{$^{1}$Albert Author is with Faculty of Electrical Engineering, Mathematics and Computer Science,
%%         University of Twente, 7500 AE Enschede, The Netherlands
%%         {\tt\small albert.author@papercept.net}}%
%% \thanks{$^{2}$Bernard D. Researcheris with the Department of Electrical Engineering, Wright State University,
%%         Dayton, OH 45435, USA
%%         {\tt\small b.d.researcher@ieee.org}}%
%% \thanks{H. Ito, M. Murooka, I. Yanokura, S. Nozawa, K. Okada and M. Inaba are with Department of Mechano-Informatics, The University of Tokyo, 7-3-1 Hongo, Bunkyo-ku, Tokyo 113-8656, Japan
%%         {\tt\small h-ito at jsk.imi.i.u-tokyo.ac.jp}}%
}


\begin{document}

\maketitle
\thispagestyle{empty}
\pagestyle{empty}

\setlength{\floatsep}{2pt} %dblfloatsep
\setlength{\textfloatsep}{2pt} %dbltextfloatsep

%\setlength{\intextsep}{4pt}
%\setlength{\abovecaptionskip}{2pt}

%\setlength{\abovedisplayskip}{5pt} % margin of top
%\setlength{\belowdisplayskip}{5pt} % margin of bottom

\begin{abstract}
\input src/abst.tex
\end{abstract}

\input src/introduction.tex
%% \input src/related_works.tex
\input src/motion_generator.tex
\input src/aural_instruction.tex
\input src/experiments.tex
%% \input src/procedure.tex
%% \input src/math.tex
%% \input src/howto.tex
\input src/conclusion.tex

\addtolength{\textheight}{-12cm}   % This command serves to balance the column lengths
                                  % on the last page of the document manually. It shortens
                                  % the textheight of the last page by a suitable amount.
                                  % This command does not take effect until the next page
                                  % so it should come on the page before the last. Make
                                  % sure that you do not shorten the textheight too much.

%% \input src/appendix.tex
%% \input src/acknowledge.tex

\bibliographystyle{junsrt}
\bibliography{main}

\end{document}


